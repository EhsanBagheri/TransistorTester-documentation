\chapter{Инструкция пользователя}
\label{sec:manual}
\section{Проведение измерений }
Использовать Тестер просто, но требуются некоторые пояснения. В большинстве случаев провода с \inquotes{крокодилами} 
подключаются к испытательным портам разъемами. Также могут быть подключены гнезда для транзисторов. В любом случае 
Вы можете подключаться тремя выводами к трем испытательным портам в любой последовательности. Если у элемента есть 
только два вывода, Вы можете подключиться к любым двум испытательным портам. Обычно полярность элемента не важна, 
Вы можете подключать выводы электролитических конденсаторов в любом порядке. Обычно минусовой вывод подключается к 
испытательному порту с более низким номером. Полярность  непринципиальна, потому что измерительное напряжение 
находится между \(0,3~V\) и \(1,3~V\). После подключения элемента, Вы не должны касаться его во время измерения. 
Если он не вставляется в гнездо, то Вы должны прижать его через непроводящую прокладку. Вы не должны также 
прикасаться к изоляции проводов, связанных с испытательными портами - результаты измерения могут быть искажены. 
После вывода на дисплей сообщения \inquotes{Testing...}, результат измерения должен появиться, примерно, после двух секунд. 
При  измерении ёмкости конденсатора время окончания  может увеличиваться пропорционально ёмкости.\\ 

Продолжительность измерения Тестера, зависит от конфигурации программного обеспечения.

\begin{description}
  \item[Режим однократного измерения.] Если Тестер сконфигурирован для однократного измерения (POWER\_OFF 
параметр установлен), то он отключается автоматически, после отображения результата в течение 28 секунд. Следующее 
измерение можно начать в течение времени отображения или после отключения, вновь нажав кнопку \textbf{ TEST}. 
Следующее измерение может быть сделано с тем же самым или другим элементом. Если Вы не установили электронные 
элементы для автоотключения, то последний результат измерения будет отображаться, пока Вы не начнете следующее 
измерение или не выключите питание (необходим внешний выключатель).

  \item[Режим бесконечных измерений.] Этот режим является конфигурацией без автоотключения.
Обычно эта конфигурация используется, если не установлен транзистор автоотключения. В этом случае, параметр POWER\_OFF 
отключается в Makefile. 
Для этого режима необходим внешний выключатель. Тестер будет повторять измерения, пока питание не будет отключено.

  \item[Режим многократных измерений.] В этом режиме Тестер отключится не после одного измерения, а после заданного 
числа измерений.
В этом случае параметру POWER\_OFF присваивается числовое значение, например 5.
В стандартном режиме Тестер отключится после 5 измерений без определения элемента. Если какой-либо 
элемент определен тестом, Тестер отключится после 10 измерений. Первое измерение с неизвестным элементом после серии 
измерений известных элементов обнулит результаты известного измерения. Также первое измерение известного элемента 
обнулит результат неизвестных измерений. Если элементы подключаются периодически, то этот алгоритм может привести к 
почти бесконечной последовательности измерений без нажатия кнопки \textbf{ TEST} в начале. В этом режиме есть характерная 
особенность длительности отображения. Если для того, чтобы включить Тестер, кнопка \textbf{ TEST} нажата коротко, 
то результат измерения отображается в течение 5 секунд. Если Вы нажимаете и держите кнопку \textbf{ TEST} до первого 
сообщения, то дальнейшие результаты измерения отображаются в течение 28 секунд. Следующее измерение можно начать 
ранее, если нажать кнопку \textbf{ TEST} во время отображения результата.

\end{description}

\section{Меню дополнительных функций для ATmega328}
Если меню дополнительных функций доступно, то оно будет отображено после продолжительного (\textgreater~\(500~ms\)) нажатия 
на кнопку \textbf{ TEST}.
Эта функция также доступна для других микроконтроллеров с объемом флэш-памяти не меньше 32K.
Выбираемые функции отображаются во второй строке дисплея в 2-х строчном LCD или, как отмеченные, в третьей строке для 4-х строчных LCD.
В 4-х строчных LCD во второй и четвертой строке отображается предыдущий и последующий пункт меню соответственно.
После длительного времени ожидания, без каких либо действий, программа выходит из меню, возвращаясь к нормальной функции транзистор тестера.  
При кратковременном нажатии на кнопку \textbf{ TEST}, осуществляется переход к следующему пункту меню. 
При длительном нажатии кнопки \textbf{ TEST} выбирается или запускается отображаемая функция меню.
После индикации последнего пункта меню \inquotes{switch off} происходит переход на первый пункт меню.\\

Если в Вашем тестере установлен поворотный энкодер, Вы можете вызывать меню дополнительных функций 
также быстрым поворотом энкодера, когда результат предыдущего теста отображается.
Функции меню можно выбрать медленным вращением энкодера в одном или другом направлении.
Выбор или запуск отображаемого пункта меню осуществляется только нажатием кнопки \textbf{ TEST}.
Параметры выбранной функции также могут быть выбраны медленным вращением энкодера.
Быстрым поворотом энкодера осуществляется возврат в меню дополнительных функций.

\begin{description}  \setlength{\itemsep}{0em}
 \item[Frequency (частотомер)]
 Дополнительная функция \inquotes{Frequency} (частотомер) использует порт PD4 ATmega, который также подключен к LCD-дисплею. 
Сначала измеряется частота. Если частота ниже \(25~kHz\), то дополнительно измеряется период входного сигнала, 
и значение этой частоты может быть вычислено с точностью до \(0,001~Hz\).
Если параметр POWER\_OFF установлен в Makefile, то продолжительность измерения частоты ограничена до 8~минут. 
Измерение частоты может быть закончено нажатием кнопки \textbf{ TEST} и Тестер перейдет в меню функций.\\

 \item[f-Generator (генератор частот)]
Если выбрана функция \inquotes{f-Generator} (генератор частоты), то можно сгенерировать любую частоту между \(1~Hz\) и \(2~MHz\).
Вы можете задавать значение генерируемой частоты только в самом старшем разряде отображаемого в строке числа.
Для старших разрядов чисел частот от \(1~Hz\) до \(10~kHz\) значения цифр изменяется от 0 до 9.
Для старших разрядов чисел задаваемой частоты выше \(100~kHz\) значения цифр изменяется от 0 до 20.
В первой позиции строки задания частоты отображается символ \textgreater~~или \textless~,
более продолжительное (\textgreater~0.8~s) нажатие кнопки \textbf{ TEST} осуществляет переход к старшему разряду
в задаваемой частоте.
Переход к младшему разряду числа задаваемой частоты, возможен при нажатии кнопки \textbf{ TEST} \textgreater~0.8~s только тогда,
когда символ \textless~отображается. Символ \textless~отображается если в старшем разряде значение цифры равно 0 и
текущая частота не ниже \(1~Hz\).
Если выбрана частота \(100~kHz\) или выше, то символ \textgreater~заменяется на букву R.
Более продолжительное (\textgreater~2~s) нажатие кнопки \textbf{ TEST} приведет к отключению генератора частоты и возврату к меню функций.\\

 \item[10-bit PWM (10-bit ШИМ)]
 Дополнительная функция \inquotes{10-bit PWM} (10-bit ШИМ) генерирует фиксированную частоту с возможностью 
регулировки ширины импульса на тестовом контакте TP2.
При кратковременном (\textless~0,5~s) нажатии кнопки \textbf{ TEST} ширина импульса увеличивается на \(1 \%\), с более длинным нажатием 
кнопки \textbf{ TEST} импульс увеличивается на \(10 \%\). 
Если значение превысило \(99 \%\), то \(100 \%\) вычитается из результата.
При установленном параметре POWER\_OFF в Makefile, генератор завершит работу после 8~минут без нажатия кнопки \textbf{ TEST}.
Завершить работу генератора можно так же длительным (\textgreater~1.3~s) нажатием кнопки \textbf{ TEST}.\\

 \item[C+ESR@TP1:3]
 Дополнительной функцией \inquotes{C+ESR@TP1:3} можно выбрать отдельное измерение ёмкости и 
ESR конденсаторов с помощью тестовых контактов TP1 и TP3.
Конденсаторы от \(2~\mu F\) до \(50~mF\) могут быть измерены. 
Поскольку напряжение измерения составляет лишь около \(300~mV\), в большинстве случаев конденсатор может быть 
измерен непосредственно в схеме без предварительного демонтажа.
При установленном параметре POWER\_OFF в Makefile, количество измерений ограничено 
до 250, но может быть начато немедленно снова.
Серия измерений может быть завершена при длительном нажатии кнопки \textbf{ TEST}.\\

 \item[Циклическое измерение сопротивлений]
Пунктом меню \mbox{1 \electricR 3} запускается циклическое измерение резисторов, подключённых к TP1 и TP3. 
Этот режим работы будет обозначен символами \textbf{[R]} справа в первой строке дисплея.
Потому что метод измерения ESR не используется в этом режиме, разрешение измерения резисторов
меньше \(10~\Omega\) только \(0.1~\Omega\).
Если функция измерения сопротивлений настроена с дополнительным измерением индуктивности, то 
символы \mbox{1 \electricR \electricL 3} будут отображаться в пункте меню. 
Тогда циклическая функция измерения сопротивлений включает проверку индуктивности для резисторов 
меньше \(2100~\Omega\).
В этом режиме справа в первой строке дисплея отобразятся символы \textbf{[RL]}.
Для резисторов меньше \(10~\Omega\) используется тот же метод измерения что и для измерения ESR,
если индуктивность не обнаружена. Таким образом, точность измерения резисторов меньше \(10~\Omega\)
может достигать значения \(0.01~\Omega\).  
Измерения повторяются без нажатия кнопки \textbf{ TEST}.
При нажатии кнопки \textbf{ TEST} осуществляется выход из режима в меню.
Циклический режим измерений запускается автоматически, если резистор подключен к TP1 и TP3 
и нажата кнопка \textbf{ TEST} из основного режима измерений. Из режима циклического измерения резисторов
тестер, по нажатию кнопки \textbf{ TEST}, возвратится в основной режим измерений.

 \item[Циклическое измерение ёмкостей]
 Пунктом меню \mbox{\begin{large}1 \electricC 3\end{large}} запускается циклическое измерение ёмкости конденсаторов в TP1 и TP3.
 Этот режим работы будет обозначен символами \textbf{[C]} справа в первой строке дисплея.
В этом режиме конденсаторы от \(1~pF\) до \(100~mF\) могу быть измерены.
Измерения повторяются без нажатия кнопки \textbf{ TEST}. 
При нажатии кнопки \textbf{ TEST} осуществляется выход из режима в меню.
Так же, как и для резисторов, циклический режим измерений запускается автоматически, если конденсатор
подключен к TP1 и TP3 в основном режиме измерений.
После автоматического запуска циклического измерения ёмкостей тестер, по нажатию кнопки \textbf{ TEST},
возвратится в основной режим измерений. 

 \item[Rotary encoder (Энкодер)]
 Дополнительная функция \inquotes{Rotary encoder} (поворотный энкодер) позволяет проверить энкодер.
Три контакта энкодера должны быть подключены в любой последовательности к тестовым контактам 
перед запуском функции.
После запуска функции нужно не слишком быстро повернуть ручку энкодера. 
Если тест завершился успешно, то во второй строке LCD будет символически отображено срабатывание 
контактов энкодера в каналах.
Тестер определяет общий вывод двух каналов. 
Кроме того, определяется состояние контактов при остановке. Если контакты замкнуты, то 
отображается \inquotes{C}, если разомкнуты, то отображается \inquotes{o}.
Результат теста энкодера с всегда разомкнутыми контактами в фиксированных позициях отображается во 
второй строке в течении двух секунд как \inquotes{1-/-2-/-3 o}.
Для такого энкодера количество фиксированных позиций соответствует количеству импульсов в каждом канале.
Если будут обнаружены замкнутые контакты каналов в фиксированных позициях, то во второй строке
дисплея отобразится \inquotes{1---2---3 C} в течении двух секунд.
Мне не известны поворотные энкодеры, которые имели бы замкнутые контакты в двух каналах одновременно
при фиксированной позиции.
Промежуточное состояние контактов каналов между фиксированными позициями также отображается в строке 2
в течении короткого времени (\textless\(~0.5s\)) без символов \inquotes{o} или \inquotes{C}.
Если Вы будете использовать энкодер для выбора в меню дополнительных функций, Вы должны установить опцию 
Makefile WITH\_ROTARY\_SWITCH=2 для энкодеров с всегда разомкнутыми контактами (\inquotes{o})  и установить опцию 
WITH\_ROTARY\_SWITCH=1 для энкодеров с разным состоянием контактов (открытым \inquotes{o} или закрытым \inquotes{C»} в 
фиксированных позициях.\\

\item[C(\(\mu F\))-correction]
С помощью этой функции меню Вы можете изменить значение поправки измерения большой ёмкости конденсаторов.
Величину этой поправки Вы можете задать параметром C\_H\_KORR в Makefile.
Значения выше нуля уменьшат величину ёмкости на заданный процент, значения ниже нуля увеличат результат
измерения ёмкости на заданный процент.
Краткое нажатие кнопки снижает значение коррекции на \(0,1~\%\), более длинное нажатие увеличивает значение коррекции.
Очень длительное нажатие сохранит значение коррекции и осуществит выход в меню.
Особенностью метода испытаний конденсаторов большой ёмкости является то, что конденсатор с низким качеством,
как электролитический, будет измерен с завышенным результатом значения ёмкости.
Конденсатор с низким качеством можно обнаружить по более высокому значению параметра VLOSS.
В конденсаторах высокого качества, при тестировании, отсутствует VLOSS или его значение не более \(0,1~\%\).
Для регулировки этого параметра Вы должны использовать только высококачественные конденсаторы ёмкостью больше \(50~\mu F\).
Кстати, я считаю, что значение точности измерения ёмкости электролитических конденсаторов неважно потому,
что значение ёмкости зависит от температуры и напряжения постоянного тока при его эксплуатации.

 \item[Selftest (Режим самотеста)]
 Дополнительная функция \inquotes{Selftest} (Режим самотеста) позволяет сделать полную самопроверку с калибровкой.
Производится самопроверка по тестам T1 - T7 и калибровка с внешним конденсатором.\\

 \item[Voltage (вольтметр)]
 Дополнительная функция \inquotes{Voltage} (вольтметр) доступна, только если отключена функции последовательного порта 
или используется ATmega с не менее чем 32 выводами (PLCC) и один из дополнительных портов ADC6 или ADC7 используется 
для измерения. 
Так как к порту PC3 (или ADC6/7) ATmega подключен делитель 10:1, то максимальное внешнее напряжение может быть не более \(50~V\).
Установленный DC-DC преобразователь для теста стабилитронов может быть включен при нажатии кнопки \textbf{ TEST}.
Таким образом, стабилитроны тоже могут быть измерены.
При установленном параметре POWER\_OFF в Makefile и без нажатия кнопки \textbf{ TEST} продолжительность измерения ограничена до 4~минут.
Измерение может быть закончено также очень длительным (\textgreater~4~s) нажатием кнопки \textbf{ TEST}.\\ 

 \item[Contrast (контрастность)]
Этой функцией можно выбрать уровень контраста для графических дисплеев с контроллерами которые поддерживают 
программную регулировку контраста.
Значение может быть уменьшено при очень коротком нажатии кнопки \textbf{ TEST} или поворотом влево энкодера.
Длительным нажатием кнопки \textbf{ TEST} (\textgreater~0,4~с) или поворотом вправо энкодера можно увеличить уровень.
Выход и запоминание выбранного значения в энергонезависимой памяти EEprom осуществляется
очень длительным нажатием кнопки \textbf{ TEST} (\textgreater~1.3~с).\\

 \item[BackColor (цвет фона)]
Для возможности выбора цвета фона в цветных дисплеях этот пункт меню необходимо активировать с помощью 
опции LCD\_CHANGE\_COLOR в makefile.
Должен быть установлен энкодер.
Вы можете выбрать красный, зеленый и синий цвет при более длительном удержании кнопки.
Интенсивность выбранного цвета, отмеченного значком > в колонке 1, может быть изменена путем поворота энкодера.\\

 \item[FrontColor (цвет выводимой информации)]
Для возможности выбора цвета выводимого шрифта и символов в цветных дисплеях этот пункт меню необходимо активировать 
с помощью опции LCD\_CHANGE\_COLOR в makefile.
Должен быть установлен энкодер.
Вы можете выбрать красный, зеленый и синий цвет при более длительном удержании кнопки.
Интенсивность выбранного цвета, отмеченного значком > в колонке 1, может быть изменена путем поворота энкодера.\\

 \item[Show data (Информация о ТТ)]
 Функция \inquotes{Show data} (Информация о ТТ), кроме номера версии программного обеспечения, показывает данные калибровки.
Нулевое сопротивление (R0) между тестовыми площадками 1: 3, 2: 3 и 1: 2, соответственно.
Кроме того, отображает сопротивление выходов портов по отношению к (\(5~V\)) напряжению питания (RiHi) и
по отношению к (\(0~V\)) GND (RiLo).
Так же показывает значения нулевой ёмкости (C0) во всех комбинациях тестовых площадок (1: 3, 2: 3, 1: 2 и 3: 1, 3: 2 2: 1).
Затем отображаются значения коррекции для компаратора (REF\_C) и для опорного напряжения (REF\_R).
Для графических дисплеев, также будут показаны все применяемые иконки и символы используемого шрифта.
Каждая страница отображается в течение 15 секунд, но, Вы можете выбрать следующую страницу нажатием кнопки \textbf{ TEST} или поворотом 
энкодера вправо.
Поворотом энкодера влево, Вы можете посмотреть последнюю страницу или вернутся на предыдущую страницу.\\

 \item[Switch off] 
 Дополнительная функция \inquotes{Switch off} позволяет выключить Тестер немедленно.\\ 

 \item[Transistor (тестер транзисторов)]
 Конечно, Вы также можете выбрать функцию \inquotes{Transistor} (тестер транзисторов), чтобы вернутся к нормальному режиму измерений Тестера.
\end{description}

При установленном параметре POWER\_OFF в Makefile, все дополнительные функции ограничены во времени, чтобы предотвратить разряд батареи.


\section{Самопроверка и калибровка}

Если программное обеспечение конфигурируется с функцей самопроверки, то самопроверка может быть запущена 
при соединении всех трёх испытательных портов вместе и нажатии кнопки \textbf{ TEST}. Чтобы начать самопроверку 
необходимо в течение 2-х секунд повторно нажать кнопку \textbf{ TEST}, иначе Тестер продолжит нормальные 
измерения.\\

Если самопроверка запущена, то будут проведены все тесты самопроверки, представленные в главе~\ref{sec:selftest}.\\

Если тестер сконфигурирован с активированным меню дополнительных функций (опция WITH\_MENU),
полная самопроверка, тест T1 - T7, выполняются только при выборе функции \inquotes{Selftest}
из меню дополнительных функций.
Кроме того, при каждом вызове функции из меню, производится калибровка с внешним конденсатором.
В противном случае эта часть калибровки делается только первый раз.
Таким образом, автоматическую калибровку можно осуществить быстрее.\\
 
Повторения тестов самопроверки можно избежать, если нажать и удерживать кнопку \textbf{ TEST}. Таким образом, Вы можете 
пропустить не интересующие Вас тесты самопроверки, и наблюдать интересующие Вас тесты самопроверки, отпуская 
кнопку \textbf{ TEST}. Тест 4 закончится автоматически, если Вы разъедините все три испытательных порта (удалите \inquotes{закоротку}).\\

Если в Makefile выбрана функция AUTO\_CAL, в режиме самопроверки будет откалибровано смещение нуля для измерения 
ёмкости. Для задачи калибровки важно, что бы \inquotes{закоротка} между тремя испытательными портами была удалена во время 
теста 4. Во время калибровки (после теста 6), Вы не должны прикасаться ни к одному из испытательных портов или 
подключенных кабелей. Шупы должны быть теми же самыми, которые будут использоваться для дальнейших измерений. 
Иначе смещение нуля для измерения ёмкости не будет правильно скомпенсировано. Величина внутреннего сопротивления 
порта определяется в начале каждого измерения с этой опцией.\\

Если Вы выбрали функцию samplingADC в Makefile опцией \inquotes{WITH\_SamplingADC = 1},
два специальных шага включаются в калибровочную процедуру.
После измерения нулевых значений ёмкости, также проводится измерение нулевых значений ёмкости для функции
samplingADC (C0samp).
В последнем шаге калибровки требуется установка конденсатора в тестовые контакты TP~1 и TP~3 для дальнейшего теста
катушек с малой индуктивностью \mbox{\begin{large}1 \electricC 3~10-30nF[L]\end{large}}. Значение ёмкости этого конденсатора должно быть
между \(10~nF\) и \(30~nF\), для получения резонансной частоты при последующем испытании индуктивностей
менее чем \(2~mH\) с параллельным подключением в контур этого же конденсатора.
Для теста катушек с более чем \(2~mH\) индуктивностью обычный метод измерения должен дать достаточную точность.
Использовать параллельное соединение конденсатора для этого метода измерения не эффективно.\\

После измерение нулевых значений ёмкости потребуется высококачественный конденсатор с любым значением между \(100~nF\) 
и \(20~\mu F\). Когда на дисплее отобразится текст \mbox{\begin{large}1 \electricC 3~\textgreater 100nF\end{large}}, Вы должны подсоединить
к испытательным выводам TP~1 и TP~3 подготовленный конденсатор. Конденсатор следует подключать не раньше, чем это 
сообщение отобразится на дисплее. С помощью этого конденсатора, будет скомпенсировано 
напряжение смещения аналогового компаратора, для более точного измерения ёмкости. Дополнительный выигрыш для 
измерений АЦП при использовании  внутреннего ИОН, с тем же самым конденсатором, дает применение опции AUTOSCALE\_ADC 
для получения лучших результатов измерения резисторов.\\

Если функция самопроверки не запрограммирована для выбора из дополнительного меню, то калибровка с внешним
конденсатором производится при первой калибровке.
Калибровку с внешним конденсатором можно повторить, выбрав соответствующий пункт дополнительного меню.\\

Смещение нуля для измерения ESR будет задано выбором опции ESR\_ZERO в Makefile. Нулевые значения ESR для всех 
трёх комбинациях выводов определяются при каждой самопроверке. Этот метод измерения ESR используется также при 
измерении величин резисторов ниже \(10~\Omega\) с разрешением \(0,01~\Omega\).


\section{Специальные возможности использования}
При включении Тестер показывает напряжение батареи питания. Если напряжение ниже предела, то после напряжения 
батареи отображается предупреждение. Если Вы используете \(9~V\) аккумулятор, то его необходимо как можно скорее 
заменить или перезарядить. Если Вы используете Тестер с \(2,5~V\) ИОН, то во второй строке в течение 1 секунды будет 
отображено напряжение питания в виде \inquotes{VCC=x.xxV}.\\

Конденсаторы должны быть разряжены перед каждым измерением. Иначе Тестер может быть повреждён еще до того, как 
будет нажата кнопка \textbf{ TEST}. При измерении элементов без демонтажа, оборудование должно быть полностью 
отключено от источника питания. Кроме того, Вы должны быть уверены, что остаточное напряжение в оборудовании 
отсутствует. У каждого электронного оборудования внутри есть конденсаторы!\\

При попытке измерить малые величины резисторов, Вы должны учитывать сопротивление разъёмов и кабелей. Очень важно 
качество и состояние разъёмов, а также, сопротивление кабелей, используемых для измерения. То же самое надо учитывать 
при измерении ESR конденсаторов. При использовании тонкого кабеля величина ESR \(0,02~\Omega\) может вырасти 
до \(0,61~\Omega\).
Если возникает необходимость в подключении испытательных щупов, то необходимо обеспечить надежное подключение
или их припаять. 
Тогда не обязательно каждый раз делать перекалибровку для измерения конденсаторов с малыми ёмкостями,
если измерения проводятся с или без измерительных щупов.
Для калибровки нулевых сопротивлений это значение имеет, если измерения проводятся с подключением выводов
непосредственно в разъеме тестера или на концах измерительных щупов.  
Только в последнем случае сопротивление кабеля и зажимов щупов будет учтено при перекалибровке.
Если у Вас есть сомнения, то Вы можете проверить сопротивление замкнутых щупов при предварительной
калибровке с использованием перемычек непосредственно в разъеме тестера.

Не стоит ожидать от Тестера высокой точности результатов, особенно при измерении ESR и индуктивности. Вы можете 
ознакомиться с результатами моей серии испытаний в главе~\ref{sec:measurement} на странице~\pageref{sec:measurement}.

\section{Проблемы при определении элементов}
Вы должны иметь в виду, интерпретируя результаты измерения, что схема Тестера разработана для слаботочных 
полупроводников. В нормальных условиях измерения измерительный ток может достигнуть приблизительно \(6~mA\). 
Мощные полупроводники часто имеют трудности с идентификацией и измерением величины ёмкости перехода из-за тока 
утечки. Тестер так же не может выдать достаточно тока для открывания или удержания  мощных тиристоров или симисторов. 
Таким образом, тиристор может быть определен как N-P-N транзистор или диод. Также возможно, что тиристор или симистор 
определятся как неизвестный элемент.\\

Другая проблема - идентификация полупроводников со встроенными резисторами. Например, диод база-эмиттер транзистора 
BU508D не может быть определен из-за параллельно подсоединённого внутреннего  резистора на \(42~\Omega\).
Поэтому параметры транзистора также не могут быть измерены. Также есть проблема с обнаружением мощных транзисторов  
Дарлингтона. Часто встречаются внутренние резисторы база - эмиттер, которые усложняют идентификацию элемента при 
малом измерительном токе.

\section{Измерение транзисторов N-P-N и P-N-P}
Для нормального измерения три вывода транзистора подключаются в любой последовательности к испытательным входам 
Тестера. После нажатия на кнопку \textbf{ TEST} Тестер показывает в первой строке тип (N-P-N или P-N-P), возможный 
встроенный защитный диод коллектор-эмиттер и последовательность выводов. Диодный символ показывается в правильной 
полярности.
Вторая строка показывает коэффициент усиления \(\beta\) или \(hFE\) и ток, при котором при котором 
этот коэффициент определен. Если используется схема измерения с общим эмиттером для определения \(hFE\),
то тестер отобразит ток коллектора \(Ic\). Если используется общий коллектор для определения коэффициента
усиления, то будет показан ток эмиттера \(Ie\).
Следующие параметры выводятся последовательно друг за другом во второй строке для двухстрочных дисплеев. 
Для дисплеев с большим количеством сток следующие параметры выводятся до заполнения последней строки.
Если для вывода всей информации строк дисплея не достаточно, то последующая информация выводится в последней 
строке чрез некоторое время или, раньше, по нажатию кнопки \textbf{ TEST}.
Если имеется больше параметров для отображения чем строк в дисплее, то символ + отображается в последней строке.  
Следующим отображается пороговое напряжение база-эмиттер. 
Если возможно измерить обратный ток коллектора при разомкнутой базе \(I_{CE0}\) и
обратный ток коллектора при замкнутых выводах базы и эмиттера \(I_{CES}\) то эти значения также будут отображены. 
Если защитный диод установлен, падение напряжения \(Uf\) будет показано, как последний параметр.
В схеме с общим эмиттером у Тестера есть только два варианта, чтобы задать базовый ток:
\begin{enumerate}
\item Резистор на \(680~\Omega\) ограничивает базовый ток приблизительно величиной \(6,1~mA\). Этот ток слишком 
велик для маломощных транзисторов с большим значением \(\beta\), потому что база насыщается. Поскольку ток коллектора 
также измеряется через резистор \(680~\Omega\) то ток коллектора не может достигнуть  величины, определяемой большим 
значением \(\beta\). Версия программного обеспечения от Markus F. измеряет пороговое напряжение база-эмиттер по этой 
схеме  (Uf=...).\\
\item Резистор на \(470~k\Omega\) ограничивает базовый ток величиной \(9,2~\mu A\).
Версия программного обеспечения от Markus F. вычисляет \(\beta\) по этой схеме (hFE =...).\\
\end{enumerate}

Программное обеспечение Тестера измеряет величину \(\beta\) дополнительно по схеме с общим коллектором. На дисплей 
выводится наибольшее значение из обоих методов измерений. Схема с общим коллектором имеет преимущество, т. к. базовый 
ток уменьшен отрицательной обратной связью, соответствующей величине \(\beta\). В большинстве случаев, более точный 
результат измерения, может быть достигнут этим методом для мощных транзисторов с резистором на \(680~\Omega\) и для 
транзисторов Дарлингтона с резистором на \(470~k\Omega\).
Пороговое напряжение  база-эмиттер Uf  теперь измеряется при том же самом токе, что и для определения 
величины \(\beta\). Однако, если Вы хотите узнать пороговое напряжение база-эмиттер с током измерения приблизительно 
\(6~mA\), то Вы должны отключить коллектор и сделать новое измерение. При этом подключении на дисплей выводится 
пороговое напряжение база-эмиттер при токе \(6~mA\).  Так же на дисплей выводится ёмкость в обратном включении перехода 
(диода). Конечно, Вы таким же образом можете проанализировать переход (диод) база-коллектор.\\

В германиевых транзисторах измеряется обратный ток коллектора при разомкнутой базе \(I_{CE0}\) и
обратный ток коллектора при короткозамкнутых выводах базы и эмиттера \(I_{CES}\). Обратный ток коллектора 
отображается во второй строке индикатора перед отображением \(\beta\) в течение 5 секунд или до следующего нажатия 
на кнопку \textbf{ TEST} (только для ATmega328).\\

При охлаждении германиевого транзистора обратный ток может уменьшиться.

\section{Измерение JFET и транзисторов D-MOS}
Поскольку структура типа JFET симметрична, исток и сток этого транзистора не могут быть определены. Обычно один из 
параметров этого транзистора - ток транзистора с затвором, на том же самом уровне напряжения, как и исток (затвор 
соединен с истоком). Этот ток часто выше, чем ток, который может быть достигнут в схеме измерения с резистором 
на \(680~\Omega\). По этой причине резистор на \(680~\Omega\) подключен к истоку. Таким образом, с ростом тока  
истока на затворе получают отрицательное напряжение смещения. Тестер показывает ток истока в этой схеме и, 
дополнительно, напряжение смещения затвора. Таким образом, могут быть выделены различные модели. Транзисторы 
D-MOS (обеднённый) измеряются тем же методом.

\section{Измерение E-MOS транзисторов и IGBT}
Вы должны знать, что для обогащенных MOS транзисторов (P-E-MOS или N-E-MOS) с малой величиной ёмкости затвора,
измерение порогового напряжения затвора (\(V_{th}\)) является более сложным.
Вы можете получить более точную величину этого напряжения, если подсоедините конденсатор величиной в несколько
\(nF\), параллельно к переходу затвор-исток.
Пороговое напряжение затвора будет измерено при токе приблизительно \(3,5~mA\) для P-E-MOS и \(4~mA\) для N-E-MOS.
RDS или, правильнее, R\textsubscript{DSon} для E-MOS транзисторов измеряется с напряжением затвора почти \(5~V\),
что, вероятно, не является самым низким значением.
Кроме того, сопротивление RDS определяется при низком токе стока, что ограничивает возможность точного 
определения значения сопротивления.
Часто в случае IGBT, а иногда и с улучшенными МОП-транзисторами доступных в тестере \(5~V\) недостаточно для 
управления транзистором через затвор.
В этом случае батарея, примерно \(3~V\), поможет сделать обнаружение и измерения с помощью тестера.
Батарея подключается к затвору транзистора одним полюсом, а другой полюс батареи подключается к тестовому порту (TP) 
вместо затвора транзистора.
Если батарея подключена с правильной полярностью, напряжение батареи добавляется к управляющему напряжению тестера,
и обнаружение транзистора более вероятно.
Значение напряжения батареи должно быть добавлено к измеренному тестером пороговому напряжению затвора, 
для получения правильного итогового порогового напряжение этого компонента.

\section{Измерение ёмкости конденсаторов}
Значения ёмкости всегда вычисляется из постоянной времени по течению операции заряда конденсатора через встроенный
резистор. Для небольших конденсаторов используются резисторы \(470~k\Omega\) при измерении времени до достижения
порогового напряжения. При тестировании больших, (\(10~\mu F\) и более) конденсаторов отслеживается время при зарядке
импульсами с резисторами \(680~\Omega\) и вычисляется ёмкость.
Совсем небольшая величина ёмкости может быть измерена с помощью метода samplingADC.
Для анализа импульс зарядки повторяется много раз, напряжение контролируется с временным сдвигом АЦП ADC S\&H 
с использованием тактов процессора. Для полного преобразования АЦП, с другой стороны нужно 1664 циклов 
процессора! До 250 значений АЦП определяются и рассчитываются от кривой напряжения 
ёмкости. Если функция samplingADC была включена в Makefile, все конденсаторы \(100~pF\) 
измеряются методом samplingADC (capacitor-meter mode \textbf{[C]} ). При тактовой частоте \(16~MHz\) 
можно получить точность до \(0,01~pF\). Процесс калибровки нулевой ёмкости представляет 
собой особую проблему. Метод определения ёмкости samplingADC применен всегда, когда Вы 
видите результат измерения ёмкости в \(pF\). Между прочим, ёмкость переходов отдельных 
диодов может быть измерена с помощью этого метода. Поскольку метод может измерять 
ёмкость и при зарядке и при разрядке, два значения измерений ёмкости отображаются. Из-за 
разной ёмкости направления переходов диода, значения различаются. 

\section{Измерение индуктивности}
Нормальное измерение индуктивности основано на измерении постоянной времени при росте тока.
Предел обнаружения составляет около \(0,01~mH\), если сопротивление катушки ниже \(24~\Omega\).
Для большего сопротивления разрешение составляет только \(0,1~mH\).
Если сопротивление выше \(2,1~k\Omega\), этот метод не может быть использован для измерения индуктивности.
Результаты измерений отображаются во второй строке (сопротивление и индуктивность).
С помощью метода samplingADC и резонансной частоты могут быть измерены катушки с большими значениями индуктивности.
Если эффект обнаружен, измеренное значение частоты и добротности Q отображается дополнительно в строке 3.

Метод измерения резонансной частотой может быть использован для определения значения индуктивности, если 
достаточно большой конденсатор с известной ёмкостью подключен параллельно малой индуктивности (\textless\(2~mH\)).

При параллельном подключении к измеряемой индуктивности конденсатора используется метод резонансной частоты, 
индуктивность нормального измерения в этом случае не отображается, а значение сопротивления отображается в строке 1.
Для этого резонансного контура добротность Q также вычисляется и её значение отображается за значением
частоты, в строке 3.
Этот тип измерения индуктивности можно определить по первой позиции в строке 2, за которым следует текст nquotas{ if }
и далее значение предполагаемой, параллельно подключенной, ёмкости.

Значение ёмкости этого параллельного конденсатора, в настоящее время, может быть задано только ёмкостью конденсатора,
который был использован во время проведения калибровки (\mbox{\begin{large}1 \electricC 3~10-30nF(L)\end{large}}).

Для дисплеев с двумя строчками, контент для третьей строки показывается с временной задержкой в строке 2.

