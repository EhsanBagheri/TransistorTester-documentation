
%\newpage
\chapter{Известные ошибки и проблемы}
{\center Версия 1.12k программного обеспечения}

\begin{enumerate}

\item Германиевые диоды (AC128) не определяются никогда. Это, вероятно, вызвано обратным током. Охлаждение диода 
может помочь уменьшить ток утечки.

\item В биполярных транзисторах защитный диод коллектор - эмиттер не может быть обнаружен, если ток ICE0 большой.
До сих пор проблема возникла только с германиевыми транзисторами с диодом не на том же кристалле.

\item Коэффициент усиления германиевых транзисторов может быть завышен из-за большого значения тока утечки. В этом 
случае напряжение база-эмиттер будет очень низким. Охлаждение транзистора может помочь получить более правильный 
коэффициент усиления.

\item Величина ёмкости в обратном направлении для мощных диодов Шоттки, таких, например, как MBR3045PT, не может 
быть измерена, если подключен только один диод. Причина - слишком большой ток утечки этого диода. Иногда измерение 
возможно при охлаждении диода.

\item Иногда выводится сообщение о неправильном обнаружении точного \(2,5~V\) ИОН, когда  порт PC4 никуда не подключен 
(вывод 27). Вы можете избежать этого поведения, установив дополнительный подтягивающий резистор на VCC.

\item Диодная функция управляющего вывода симистора не может быть исследована.

\item Иногда происходит сброс во время измерения ёмкости, что говорит о проблеме с Brown Out Level \(4,3~V\) для 
ATmega168 или ATmega328. Причина не известна. Сброс исчезает, если Brown Out Level установить на \(2,7~V\).

\item При использовании SLEEP MODE микроконтроллера ток питания VCC изменяется больше, чем при использовании 
предыдущих версий программного обеспечения. Вы должны увеличить блокировочные конденсаторы, если замечаете 
какие-либо проблемы. Керамические конденсаторы \(100~nF\) должны быть помещены около выводов питания ATmega. 
Использование SLEEP MODE можно отключить опцией INHIBIT\_SLEEP\_MODE в Makefile.

\item Часто не измеряются танталовые электролитические конденсаторы. Они могут быть обнаружены, как диод или 
могут быть не обнаружены вообще. Иногда помогает отключение-подключение.

\item Выводы \inquotes{исток} и \inquotes{сток} не могут быть определены корректно в JFET транзисторах.
Причина в симметричности их структуры.
Вы можете заметить эту проблему, поменяв местами в тестовых контактах «сток» и «исток», а на дисплее отобразится
предыдущее расположение выводов. Я не вижу никакой возможности корректно определить выводы этих транзисторов.
Но, перестановка местами \inquotes{сток} и \inquotes{исток} в схемах, как правило, не вызывают каких либо проблем.   

\end{enumerate}
