\section*{Preface}
\subsection*{Basic Motive}
Every hobbyist knows the following problem: You remove a transistor from a printed circuit board (PCB)
or you get one out of a collection box. 
If you find out the identification number and you already have a data sheet or you can get the documents about this part,
 everything is well.
But if you don't find any documents, you have no idea what kind of part this could be.
With the conventional approach of measurement it is difficult and time-consuming to find out the type of the part and its parameters.
It could be a NPN, PNP, N- or P-channel MOSFET, etc.
It was the idea of Markus F. to hand over the work to an AVR microcontroller.
\subsection*{How my work started}
My work with the software of the TransistorTester of Markus F. \cite{Frejek} started because I had problems with 
my programmer. I had bought a PCB and components, but I could not program the EEprom of 
the ATmega8 using the Windows driver without error messages. Therefore I took the software of Markus F. and changed all the accesses
from the EEprom memory to flash memory accesses. By analysing the software in order to save memory
at other places of program, I had the idea, to change the result of the ReadADC function from ADC units
to millivolt (mV) units. The mV units are needed for any output of voltage values.
If ReadADC returns the value in mV, I can avoid unnecessary repeated conversion of returned value.
To obtain the ADC value in mV, we first accumulate the sum of 22 ADC readings.
 The sum is multiplied by two and divided by nine. That gives a maximum value of \begin{math}\frac{1023\cdot22\cdot2}{9} = 5001\end{math},
which matches perfectly the desired mV scale of measured voltage values.
So I additionally had the hope that the enhancement of ADC resolution by oversampling could help to improve the
voltage reading of the ADC, as described in AVR121 \cite{AVR121}.
The original version ReadADC accumulated the result of 20 ADC measurements and divided afterwards by 20, 
so the result is equal to original ADC resolution. This was suboptimal because it missed an opportunity to enhance the ADC's resolution.
Changing ReadADC was a small amount of work, but this forced analysing the whole program and changing
all conditional statements in the program where voltage values are queried.
But this was only the beginning of my work!\\

More and more ideas to make measurement faster and more accurate were implemented.
Additionally the range of resistor and capacity measurements were extended.
The output format for the LCD display was changed, using symbols for diodes, resistors and capacitors instead of text.
For further details take a look to the actual feature list chapter~\ref{sec:features}.
Planned work and new ideas are accumulated in the To Do List in chapter~\ref{sec:todo}.
By the way, now I can program the EEprom of the ATmega with Linux operating system without errors.

At this point I would like to thank the originator and software author, Markus Frejek, who has enabled the continuation
of his initial work.
In addition I would like to say thanks to the providers of input on the discussion forum. This has helped me to
find new tasks, weak points and errors.
Next I would like to thank Markus Reschke, who give me the permission, to publish his cheerful software versions on the
SVN server. Furthermore some ideas and software by Markus R. was integrated in my own software version.
Again, thank you very much.
Also Wolfgang SCH. has done a great job of supporting a graphical display with the ST7565 controller. Many thanks to him
for integrating his patch for 1.10k software version to the actual developer version.
I have to thank also Asco B., who has developed a new printed board, to enable the reproductions for other hobbyists.
Another thanks I would like to send is to Dirk W. , who has handled the omnibus order for this printed board.
I didn't have time to handle these things concurrently with my software developement. This assistance helped push the
software develpment forward.
Thanks for the many suggestions to improve the tester from the members of the local chapter of the \inquotes{Deutscher Amateur Radio Club (DARC)}
in Lennestadt.
Also I would like to say thanks for the integration of the sampling method by the radio amateur Pieter-Tjerk (PA3FWM).
Using this method, the measurement of small capacitance and inductance values was significantly imrpoved.
Last but not least I would say thanks to Nick L. from Ukraine, who has supported me with his prototype boards, 
has suggested some hardware add-ons and also has organized the Russian translation for this description.

